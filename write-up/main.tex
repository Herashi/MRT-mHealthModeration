%\documentclass[manuscript]{biometrika}
\documentclass[lineno]{biometrika}


\usepackage{amsmath}

%% Please use the following statements for
%% managing the text and math fonts for your papers:
\usepackage{times}
%\usepackage[cmbold]{mathtime}
\usepackage{bm}
\usepackage{natbib}

\graphicspath{{./art/}}
\usepackage{booktabs}
\usepackage{amsfonts}
\usepackage{multirow}
\usepackage{arydshln}
% \usepackage{caption}

\usepackage[plain,noend]{algorithm2e}

\makeatletter
\renewcommand{\algocf@captiontext}[2]{#1\algocf@typo. \AlCapFnt{}#2} % text of caption
\renewcommand{\AlTitleFnt}[1]{#1\unskip}% default definition
\def\@algocf@capt@plain{top}
\renewcommand{\algocf@makecaption}[2]{%
  \addtolength{\hsize}{\algomargin}%
  \sbox\@tempboxa{\algocf@captiontext{#1}{#2}}%
  \ifdim\wd\@tempboxa >\hsize%     % if caption is longer than a line
    \hskip .5\algomargin%
    \parbox[t]{\hsize}{\algocf@captiontext{#1}{#2}}% then caption is not centered
  \else%
    \global\@minipagefalse%
    \hbox to\hsize{\box\@tempboxa}% else caption is centered
  \fi%
  \addtolength{\hsize}{-\algomargin}%
}
\makeatother

\def\pr{\text{pr}}
\def\E{\mathbb{E}}
\def\P{\mathbb{P}}
\def\given{\, | \,}
\def\Ep{\E_{\bf p}}
\def\Eeta{\E_{\eta}}
\def\U{\mathcal{U}}
\def\V{\mathcal{V}}

%%% User-defined macros should be placed here, but keep them to a minimum.
\def\Bka{{\it Biometrika}}
\def\AIC{\textsc{aic}}
\def\T{{ \mathrm{\scriptscriptstyle T} }}
\def\v{{\varepsilon}}

\addtolength\topmargin{35pt}
\DeclareMathOperator{\Thetabb}{\mathcal{C}}


\usepackage{xr}
\makeatletter
\newcommand*{\addFileDependency}[1]{
  \typeout{(#1)}
  \@addtofilelist{#1}
  \IfFileExists{#1}{}{\typeout{No file #1.}}
}
\makeatother

\newcommand*{\myexternaldocument}[1]{
    \externaldocument{#1}
    \addFileDependency{#1.tex}
    \addFileDependency{#1.aux}
}
%%% END HELPER CODE

% put all the external documents here!
\myexternaldocument{appendix}



\begin{document}

% \jname{}
% %% The year, volume, and number are determined on publication
% \jyear{}
% \jvol{}
% \jnum{}
% %% The \doi{...} and \accessdate commands are used by the production team
% %\doi{10.1093/biomet/asm023}
% \accessdate{}

% %% These dates are usually set by the production team
% \received{}
% \revised{}

%% The left and right page headers are defined here:
\markboth{J. Shi, Z. Wu, \and W. Dempsey}{Causal Excursion Effects and Treatment Effect Heterogeneity}

%% Here are the title, author names and addresses
\title{Assessing Time-Varying Causal Effect Moderation in the Presence of Cluster-Level Treatment Effect Heterogeneity}

\author{J. SHI, Z. WU, \and W. Dempsey}
\affil{Department of Biostatistics, University of Michigan, \\ Ann Arbor, MI, USA \email{herashi@umich.edu} \email{zhenkewu@umich.edu} \email{wdem@umich.edu}}

\maketitle

\begin{abstract}
The micro-randomized trial (MRT) is a sequential randomized experimental design to empirically evaluate the effectiveness of mobile health (mHealth) intervention components that may be delivered at hundreds or thousands of decision points. MRTs have motivated a new class of causal estimands, termed ``causal excursion effects", for which semiparametric inference can be conducted via  a weighted, centered least squares criterion (Boruvka et al., 2018). Existing methods assume between-subject independence and non-interference. Deviations from these assumptions often occur.
% which, if unaccounted for, may result in bias and overconfident variance estimates. 
In this paper, causal excursion effects are considered under potential cluster-level treatment effect heterogeneity and interference, where the treatment effect of interest may depend on cluster-level moderators. Utility of the proposed methods is shown by analyzing data from a multi-institution cohort of first year medical residents in the United States. 
\end{abstract}

\begin{keywords}
Causal Inference; Clustered Data; Just-In-Time Adaptive Interventions; Microrandomized Trials; Mobile Health; Moderation Effect
\end{keywords}

\section{Introduction}

Modern behavioral science has placed a considerable amount of attention on push notifications sent via mobile device that are adapted to continuously collected information on an individual's current context.  These time-varying adaptive interventions are hypothesized to lead to meaningful short- and long-term behavior change. The assessment of the time-varying effect of such push notifications motivated sequential randomized designs such as the micro-randomized trial (MRT) \citep{Nahum2017, KlasnjaMRT}, in which individuals are randomized to potentially receive notifications at hundreds or thousands of decision points. The MRT design enables the estimation of marginal treatment effects of push notifications on pre-specified time-lagged outcomes of interest, referred to as ``causal excursion effects" \citep{Boruvkaetal, Qian2021, DempseyAOAS}. Semiparametric inference of the causal excursion effects can be conducted via a weighted, centered least squares (WCLS) criterion \citep{Boruvkaetal}. 
%The WCLS approach has also been used in analyzing data obtained from stratified MRT~\citep{DempseyAOAS}.

The WCLS inferential method relies on two key assumptions. First, an intervention delivered to an individual is assumed to only impact that individuals'  outcomes, i.e., between-subject non-interference. Second, the method assumes no stochastic dependence among outcomes of different subjects. Deviations from these assumptions, however, may occur when individuals naturally form clusters. To address these violations, this paper extends the definition of causal excursion effects to account for potential interference and treatment effect heterogeneity. We propose and illustrate a novel inferential approach that ensures proper interval coverage. 

\section{Preliminaries}
\label{section:preliminaries}

\subsection{Micro-Randomized Trials (MRT)}

An MRT consists of a sequence of within-subject decision times $t=1,\ldots,T$ at which treatment options may be randomly assigned~\citep{Liaoetal2015}.  Individual-level data can be summarized as~$\{ O_0, O_1, A_1, O_2, A_2, \ldots, O_T, A_T, O_{T+1} \}$
where $t$ indexes a sequence of decision points, $O_0$ is the baseline information, $O_t$ is the information collected between time $t-1$ and $t$, and $A_t$ is the treatment option provided at time $t$; for simplicity, we consider binary treatment options, i.e.,~$A_t \in \{ 0, 1\}$.  In an MRT, $A_t$ is randomized according to a known sequence of randomization probabilities that may depend on the complete observed history $H_t := \{ O_0, O_1, A_1, \ldots, A_{t-1}, O_t \}$,  denoted ${\bf p} = \{ p_u (A_u \given H_u) \}_{u=1}^t$. Treatment options are designed to impact a proximal response, denoted by $Y_{t,\Delta}$, which is a known function of the participant’s data within a subsequent window of length~$\Delta \ge 1$~\citep{DempseyAOAS}.

\subsection{Estimand and Inferential Method: A Review}
\label{section:standardmrtmethods}

% NEED CLEANER INTRO INTO DELTA ISSUE
% COULD DO POTENTIAL OUTCOME QUICKLY?
% In this paper, we focus on presenting the causal analysis with time lag $\Delta=1$; the framework is general and applies to an arbitrary number of lags $\Delta \geq 1$ \citep{Boruvkaetal}. This section assumes stochastic independence between subjects and briefly reviews the existing estimand and inferential precedure. 
We focus on the class of estimands referred to as ``causal excursion effects", which are time-varying as a function of the decision point~$t$, and which are formally defined using potential outcomes~\citep{Rubin, Robins}.  Let~$Y_{t,\Delta} (\bar a_{t+\Delta-1})$ denote the potential outcome for the proximal response under treatment sequence~$\bar a_{t+\Delta-1} = (a_1, \ldots, a_{t+\Delta-1})$.  Let $S_t (\bar a_{t-1})$ denote the potential outcome for a potential time-varying effect moderator which is a deterministic function of the potential history up to time $t$, $H_t (\bar a_{t-1})$. The causal excursion effect is then defined with respect to a \emph{reference distribution}, i.e., the distribution of treatments up to time~$t+\Delta-1$. We follow common practice in observational mobile health studies where analyses such as GEEs~\citep{Liang1986} are conducted marginally over the distribution of historical information.  A similar strategy here is to use the past treatment randomization probabilities as the reference distribution.  The choice of distribution for $a_{t+1:(t+\Delta-1)} := (a_{t+1}, \ldots, a_{t+\Delta-1})$ will differ by the type of inference desired.  Here, we assume the reference distribution for treatment assignments from $t+1$ to $t+\Delta-1$ is given by a randomization probability generically represented by~$\pi_{u}(a_{u} | H_{u}), u=t+1,\ldots, t+\Delta-1$ and let $\pi=\{\pi_{u}\}_{u=t+1}^{t+\Delta-1}$.  This generalization contains previous definitions such as lagged effects~\citep{Boruvkaetal} where $\pi_{u} = p_{u}$ and deterministic choices such as $a_{t+1:(t+\Delta-1)} = {\bf 0}$~\citep{DempseyAOAS, Qian2021} where $\pi_{u} = {\bf 1}\{a_{u} = 0\}$ and $1\{\bullet\}$ is the indicator function.  Then the causal excursion effect $\beta_{{\bf p}, \pi, \Delta} (t;s)$ is defined as

\begin{align}
&\E_{{\bf p}, \pi} \left [ Y_{t,\Delta} \left(\bar A_{t-1}, 1, \tilde A_{t+1:(t+\Delta-1)} \right) - Y_{t,\Delta} \left(\bar A_{t-1}, 0, \tilde A_{t+1:(t+\Delta-1)} \right) \given S_t (\bar a_{t-1}) = s\right] \label{eq:causalexcursion_po} \\
=  &\E_{{\bf p}} \left[ \E_{{\bf p}} \left[ W_{t,\Delta} Y_{t,\Delta} \mid A_t = 1, H_t \right] - \E_{{\bf p}} \left[ W_{t,\Delta} Y_{t,\Delta} \mid A_t = 0, H_t \right] \mid S_t = s \right] \label{eq:causalexursion}
\end{align}
where treatment sequence up to time $t-1$: $\bar A_{t-1} \sim {\bf p}$, future treatment sequence up to $t+\Delta-1$: $\tilde A_{t+1:(t+\Delta-1)} \sim \pi$ and~$W_{t,\Delta} = \prod_{u=t+1}^{t+\Delta-1} \pi_u (A_u | H_u) / p_u(A_u | H_u)$. 
Equation~\eqref{eq:causalexursion} expresses \eqref{eq:causalexcursion_po} in terms of observable data, which requires the standard causal inference assumptions of positivity, sequential ignorability, and consistency.  
Assuming $\beta_{{\bf p}, \pi, \Delta} (t;s) = f_t(s)^\top \beta^\star$ where $f_t(s) \in \mathbb{R}^q$ is a feature vector comprised of a $q$-dimensional summary of observed state information depending only on state $s$ and decision point $t$, a consistent estimator for $\beta^*$ can be obtained by minimizing a weighted and centered least squares (WCLS) criterion:
\begin{equation}
\label{eq:mrtstandard}
\hat \beta = \arg \min_{\alpha, \beta} \mathbb{P}_n \left[ \sum_{t=1}^T W_t \times W_{t,\Delta} \left[ Y_{t,\Delta} - g_t(H_t)^\top \alpha - \left ( A_t - \tilde p_t (1 \mid S_t) \right) f_t (S_t)^\top \beta \right]^2 \right]
\end{equation}
where~$\mathbb{P}_n$ is shorthand for the sample average, $W_t = \tilde p_t (A_t \mid S_t) / p_t (A_t \mid H_t)$ is a weight where the numerator is an arbitrary function with range $(0,1)$ that only depends on $S_t$, and $g_t(H_t) \in \mathbb{R}^p$ are $p$ control variables chosen to help reduce variance and to construct more powerful test statistics. See \citet{Boruvkaetal} for more details on the seminal estimand formulation and WCLS estimation method.
% In HeartSteps, for example, a natural control variable is the number of steps in the prior 30 minutes which is likely highly correlated with the proximal response and thus can be used to reduce variance in the estimation of $\beta$. See~\cite{Boruvkaetal} for estimation, consistency, and small-sample corrected uncertainty assessment that is robust to the independent working correlation assumption in~\eqref{eq:mrtstandard}.

\subsection{Motivating Example}
\label{section:motex}

The Intern Health Study (IHS) is a 6-month MRT on 1,562 medical interns~\citep{Necamp2020}.  Due to high depression rates and levels of stress during the first year of physician residency training, a critical question is whether targeted notifications can improve mood, increase sleep time, and/or increase physical activity. Enrolled medical interns were randomized weekly to receive either mood, activity, or sleep notifications or receive no notifications for that week (probability 1/4 each).  Analyses conducted in this paper focus on the weekly randomization; see~\cite{Necamp2020} for further study details.  Figure~\ref{fig:wcls_heterogeneity} presents effect estimates per specialty on weekly average mood scores using~\eqref{eq:mrtstandard} which shows potential specialty-level treatment effect heterogeneity.  This suggests a marginal analysis must account for effect heterogeneity at the specialty-level. The present work offers such a framework and demonstrate the standard WCLS cannot account for effect heterogeneity. In addition, there exists potential within-cluster interference of other subjects' treatments upon a subject's outcome. Our framework defines an indirect excursion effect under sequential treatments in contrast to existing work that mostly focuses on indirect effects in non-temporal settings.

\begin{figure}
  \figuresize{.3}
  \figurebox{10pc}{15pc}{}[heterogeneityv2.eps]
  \caption{Observed treatment effect heterogeneity of the estimated causal excursion effect estimated using the existing WCLS across the largest 19 clusters (i.e., specialties with greater than or equal to 6 people).}
  \label{fig:wcls_heterogeneity}
\end{figure}

\section{Cluster-Level Proximal Treatment Effects}
\label{section:cond_effects}

\subsection{Proximal Moderated Treatment Effects: A Cluster-based Conceptualization}
\label{section:prox_effects_pot_outcome}

Consider a cluster of size $G$.  Overbar will continue to denote treatment sequences; $\bar a_{t,j} = (a_{1,j},\ldots, a_{t,j})$, for instance, denotes the sequence of realized treatment sequence up to and including decision time $t$ for individual $j \in [G]:=\{1,\ldots, G\}$.   Let $\bar a_{t} = (\bar a_{t,1}, \ldots, \bar a_{t,G})$ denote the set of realized treatments for all individuals in the cluster. Let $\bar a_{t,-j} = \bar a_t \backslash \bar a_{t,j}$ denote this set with the $j$th individual removed. Let $Y_{t,\Delta,j} (\bar a_{t+\Delta-1})$ denote the potential outcome for individual $j \in [G]$ which may depend on realized treatments for all subjects in the cluster.
% and that cluster-size may vary.
% \zw{cluster sizes may differ}

\noindent \textbf{Direct causal excursion effects}.
In standard MRTs, the individual is the unit of interest.  Here, the cluster is the unit of interest and the effect of interest is in providing treatment versus not providing treatment at time $t$ on a random individual in the group.  This can be expressed as a difference in potential outcomes for the proximal response
\begin{equation}
\label{eq:directeffect}
\begin{array}{r@{}l}
\frac{1}{G} \sum_{j=1}^G \bigg [ Y_{t, \Delta,j} &(\bar a_{t+\Delta-1,-j}, (\bar a_{t-1,j}, 1, a_{t+1:(t+\Delta-1),j})) \\
&- Y_{t,\Delta, j} (\bar a_{t+\Delta-1,-j}, (\bar a_{t-1,j}, 0, a_{t+1:(t+\Delta-1),j})) \bigg].
\end{array}
\end{equation}
Following~\cite{Halloran1995} and~\cite{Tchetgen2012}, \eqref{eq:directeffect} is a \emph{group average direct causal effect} of treatment versus no treatment fixing all other treatments. 
% \zw{clarify the directness means blocking all other paths except the path from $A_{tj}$ to $Y_{t+1,j}$}

The ``fundamental problem of causal inference''~\citep{Rubin, Pearl2009} is that individual differences cannot be observed. Thus, similar to prior work~\citep{DempseyAOAS, Boruvkaetal}, averages of potential outcomes are considered. Let $S_t (\bar a_{t-1})$ denote a vector of potential moderator variables chosen from $H_t (\bar a_{t-1})$, the \emph{cluster-level} history up to decision point $t$.  
% We write $S_t (\bar a_{t-1}) = \left( S_{t,j} (\bar a_{t-1}), S_{t,-j} (\bar a_{t-1}) \right)$ to clarify that the potential moderator variables can contain both information on the selected individual as well as other individuals in the cluster. 
Then the moderated direct treatment effect, denoted $\beta_{{\bf p}, \pi, \Delta} (t; s)$, can be defined as
\begin{equation}
\label{eq:directavglineareffect}
\begin{array}{r@{}l}
\mathbb{E}_{{\bf p}, \pi} \bigg[ &Y_{t,\Delta,J} (\bar A_{t+\Delta-1,-J}, (\bar A_{t-1,J}, 1, \tilde A_{t+1:(t+\Delta-1),J} )) \\
- &Y_{t,\Delta,J} (\bar A_{t+\Delta-1,-J}, (\bar A_{t-1,J}, 0,\tilde A_{t+1:(t+\Delta-1), J})) \mid  S_{t} (\bar A_{t-1}) = s \bigg].
\end{array}
\end{equation}
where $J$ is a uniformly distributed random index defined on $[G]$. The expectation is over the potential outcomes~$Y_{t,\Delta,J}(\cdot)$, the randomized treatments -- $\bar A_{t+\Delta-1,-J} \sim {\bf p}$, $\bar A_{t-1,J} \sim {\bf p}$, and $\tilde A_{t+1:(t+\Delta-1),J} \sim \pi$ -- and the random index~$J$.
% \zw{Perhaps can talk about the expectation is over both Y and A; and write it in a way that addresses potential issues that may be raised by the utility of this estimand and how is it different from existing ones.}  \wd{Now raising the expectations point but pushing discussion of the estimand and its difference from other stuff to that 3.2.3 section}
Choice of $S_t (\bar A_{t-1})$ depends on the scientific question of interest.  A primary analysis may focus on marginal effects and set $S_t (\bar A_{t-1}) = \emptyset$.  A second analysis may focus on assessing the effect conditional on variables only related to the individual indexed by $J$ and set $S_t (\bar A_{t-1}) = X_{t,J} (\bar A_{t-1,J})$, i.e., a potential individual-level moderator of interest.  A third analysis may consider group-level moderators such as $S_t (\bar A_{t-1}) = G^{-1} \sum_j X_{t,j} (\bar A_{t-1,j})$ or $S_t (\bar A_{t-1}) = \left( X_{t,j}(\bar A_{t-1,j}), \frac{1}{G-1} \sum_{j^\prime \neq j} X_{t,j^\prime} (\bar A_{t-1,j^\prime}) \right)$. Equation~\eqref{eq:directavglineareffect} generalizes the \emph{population average direct causal effect} from~\cite{Tchetgen2012} to a \emph{group-level causal excursion effect} that allows for moderation and time-varying treatments. 

\noindent \textbf{Pairwise indirect causal excursion effects}.
Of secondary interest is the indirect effect of providing treatment versus not providing treatment to the $j$th individual at time $t$ on a different individual's proximal response, i.e., pairwise within-cluster treatment interference.  Here, we define the  \emph{pairwise indirect causal excursion effect} as
\begin{align}
\frac{1}{G \cdot (G-1)} &\sum_{j^\prime \neq j} \left[Y_{t,\Delta,j} (\bar a_{t+\Delta-1,-\{j, j^\prime\}}, (\bar a_{t-1,j}, 0, a_{t+1:(t+\Delta-1),j}), (\bar a_{t-1,j^\prime}, 1, a_{t+1:(t+\Delta-1),j'})) \right.\nonumber\\
&- \left.Y_{t,\Delta,j} (\bar a_{t+\Delta-1,-\{j, j^\prime\}}, (\bar a_{t-1,j}, 0, a_{t+1:(t+\Delta-1),j}), (\bar a_{t-1,j^\prime}, 0, a_{t+1:(t+\Delta-1),j'})) \nonumber \right].
\end{align}
Again, since individual differences cannot be observed, averages of potential outcomes are considered. The moderated pairwise indirect treatment effect, denoted $\beta^{(IE)}_{{\bf p}, \pi, \Delta} (t; s)$, is
\begin{equation}
\label{eq:inddirectavglineareffect}
\begin{array}{r@{}l}
&\mathbb{E}_{{\bf p}, \pi} \bigg[ Y_{t,\Delta,J} \left(\bar A_{t+\Delta-1,-\{J, J^\prime\}}, (\bar A_{t-1,J}, 0, \tilde A_{t+1:(t+\Delta-1),J}), (\bar A_{t-1,J'}, 1, \tilde A_{t+1:(t+\Delta-1),J'}) \right) \\
&- Y_{t,\Delta,J} \left(\bar A_{t+\Delta-1,-\{J, J^\prime\}}, (\bar A_{t-1,J}, 0, \tilde A_{t+1:(t+\Delta-1),J}), (\bar A_{t-1,J'}, 0, \tilde A_{t+1:(t+\Delta-1),J'}) \right) \mid S_t (\bar A_{t-1}) \bigg].
\end{array}
\end{equation}
where $J^\prime$ is uniformly distributed random index on the set $[G] \backslash \{J\}$. The expectation is again over both the potential outcomes~$Y_{t,\Delta,J}(\cdot)$, randomized treatments -- $\bar A_{t+\Delta-1,-\{J,J^\prime\}} \sim {\bf p}$, $\bar A_{t-1,J} \sim {\bf p}$, $\bar A_{t-1,J^\prime} \sim {\bf p}$, $\tilde A_{t+1:(t+\Delta-1),J} \sim \pi$, and $\tilde A_{t+1:(t+\Delta-1),J^\prime} \sim \pi$ --  and the random indices~($J$ and $J^\prime$). Here, the potential moderator can be written as $S_t (\bar A_{t-1}) = \left( S_{t,J} (\bar A_{t-1}), S_{t,J^\prime} (\bar A_{t-1}), S_{t,-\{J, J^\prime \}} (\bar A_{t-1}) \right)$ to clarify that the variables can contain both information on the two selected individuals as well as others in the cluster. Another pairwise indirect effect can be defined when individual~$J$ receives treatment, i.e., $A_{t,J} = 1$ instead of~$A_{t,J} = 0$ as in~\eqref{eq:inddirectavglineareffect}.

\begin{remark}
The effect defined by~\eqref{eq:inddirectavglineareffect} generalizes the \emph{group average indirect causal effect} from~\cite{Tchetgen2012} to a \emph{group-level causal excursion effect} that allows for moderation and time-varying treatments. To see this, note that the excursion effect at each decision time~$t$ averages over a particular reference distribution over the past and future treatments up to and including time~$t+\Delta-1$ defined by the MRT randomization probabilities~$\bf p$ and the alternative probability distribution~$\pi$.  The contrast is over two treatment allocations, both where a random individual does not receive treatment, but where in one allocation another random individual receives treatment and in the other allocation that same individual does not receive treatment. \cite{Tchetgen2012} consider contrasts between any two randomized treatment allocations conditional on a random individual not receiving treatment in a non-temporal setting.  Therefore, per decision time our definition is a special case of their indirect effect where we focus on treating or not treating one individual and marginalizing over all others in the group.  More complex contrasts could be derived such as three- or four-way indirect effects; however, the number of combinations grows quickly making estimation unrealistic in our setting. Our choice of contrast was thus motivated by finding an estimand of scientific interest which could be reasonably estimated within the MRT setting, bridging the literature on causal excursions and indirect effects.  
% so~\eqref{eq:inddirectavglineareffect} can be seen as a generalization to the time-varying treatment setting. 
\end{remark}

\subsection{Causal Excursion Effect Estimand Depends on Treatment Distribution}

% It is useful to distinguish the above estimands, the traditional MRT estimand~\citep{Boruvkaetal}, and estimands commonly studies in longitudinal treatment effect estimation~\citep{Robins}. In the causal inference literature, the typical estimand is an expected outcome for a particular sequence of treatments,i.e.,~$\E \left[ Y(a_1,\ldots, a_T)\right]$.  Such estimands do not depend on the treatment distribution, but are often not of primary interest in the current MRT setting since many sequences of treatment may never be observed in finite samples given the large number of decision points (often hundreds or thousands).

Estimands considered here are most similar to average outcomes under a particular dynamic treatment regime~$\E_\mu \left[ Y(A_1,\ldots, A_T)\right]$ where $\mu$ denotes the dynamic treatment regime from which the treatments are drawn~\citep{MurphyJASA2001}. Indeed, for any $A_{u,j}$ not contained in $S_t (\bar A_{t-1})$, the direct and indirect effects depend on the distribution of $\{ A_{u,j} \}_{u < t + \Delta -1, j \in [G]}$.  Estimands~\eqref{eq:directavglineareffect} and~\eqref{eq:inddirectavglineareffect} marginalize over treatments not contained in $S_t(\bar A_{t-1})$.  Marginalization over different probabilistic assignment of treatments may yield different results. Therefore, the direct and indirect excursion effects depend on the study protocol and choice of alternative distribution~$\pi$. The reason for this is that micro-randomization is meant to gather information on how to optimize the design of intervention components~\citep{Collins2018}.  The marginal formulation of main and moderation effects contrasts excursions from the current treatment protocol, and mimics analyses used in a factorial design that marginalize over factors including time. See~\citet[Section 8]{Qian2021} for additional considerations.  Regardless, the effects considered in this paper are causal and depend on the treatment assignment distributions.  Due to this dependence, in real data analysis, we recommend presenting the micro-randomization distribution together with the estimated treatment effects, thus the subscript~$({\bf p}, \pi)$ in the definition of both the direct and indirect effects are specified.

\subsection{Identification}
\label{section:prox_effects_data}
Causal effects \eqref{eq:directavglineareffect} and~\eqref{eq:inddirectavglineareffect} can be expressed in terms of the observable data under the following standard set of causal inference assumptions~\citep{Robins}:

\begin{assumption} \normalfont
  \label{consistency}
  We assume consistency, positivity, and sequential ignorability:
  \begin{itemize}
  \item Consistency: For each~$t \leq T$ and $j \in [G]$,
    $\{Y_{t,\Delta,j} (\bar{A}_{t+\Delta-1} ), O_{t,j} (\bar A_{t-1}), A_{t,j} (\bar{A}_{t-1} )\}  = \{Y_{t, \Delta, j}, O_{t,j}, A_{t,j}\}$, i.e., observed values equal the corresponding potential outcomes;
  \item Positivity: if the joint density~$\{ A_t = a_t, H_t = h_t\}$ is greater
    than zero, then~$P (A_t = a_t \given H_t = h_t ) > 0$;
  \item Sequential ignorability: for each~$t \leq T$, the
    potential outcomes,\\ $\{Y_{2,\Delta,j} ( \bar a_{1+\Delta-1}), O_{2,j}(a_{1}),A_{2,j}( a_{1}), \ldots,
    Y_{T,\Delta, j} (\bar a_{T+\Delta-1}) \}_{j \in [G], \bar a_{T+\Delta-1}\in \{0,1\}^{(T+\Delta-1)\times G}}$, are independent of~$A_{t,j}$ conditional on the observed history~$H_t$.
  \end{itemize}
\end{assumption}

Sequential ignorability and, assuming all of the randomization probabilities are bounded away from $0$ and $1$, positivity, are guaranteed in our setting  by design. 
% In a standard MRT, the randomization probabilities may depend on the individual's observed history so $\Pr(A_t = a_t \mid H_t = h_t) = \prod_{j=1}^G \Pr(A_{t,j} = a_{t,j} \mid H_{t,j} = h_{t,j})$ and the positivity constraint can be placed on the individual-level randomization probabilities.  Here, we allow for the possibility that the randomization probabilities depend on the cluster-level history. 
Consistency is a necessary assumption for linking the potential outcomes as defined here to the data. Since an individual's outcomes may be influenced by the treatments provided to other individuals in the same cluster, consistency holds due to our use of a cluster-based conceptualization of potential outcomes as seen in~\cite{Hong2006} and \cite{Vanderweele2013}.

\begin{lemma}
\label{lemma:cond_effect}
  Under Assumption~\ref{consistency}, the moderated direct treatment effect~$\beta_{{\bf p}, \pi, \Delta} (t;s)$ is equal to
  $$
  \mathbb{E} \left[ \mathbb{E} \left[ W_{t, \Delta, J} Y_{t,\Delta,J} \mid H_{t}, A_{t,J} = 1 \right] - \mathbb{E} \left[ W_{t, \Delta, J} Y_{t, \Delta,J} \mid H_{t}, A_{t,J} = 0 \right] \mid S_t=s \right],
  $$
  where expectations are with respect to the distribution of the
  data collected under the treatment assignment probabilities and~$W_{t,\Delta,j} = \prod_{u=t+1}^{t+\Delta-1} \pi(A_{u,j} | H_u) / p(A_{u,j} | H_u)$; and the moderated indirect treatment effect~$\beta^{(IE)}_{{\bf p}, \pi, \Delta}(t;s) $ is equal to
  \begin{align*}
  \mathbb{E} \bigg[ \mathbb{E} &\left[ W_{t,\Delta,J,J'} Y_{t,\Delta,J} \mid H_{t}, A_{t,J} = 0, A_{t,J^\prime} = 1 \right] \\
  - &\mathbb{E} \left[ W_{t,\Delta,J,J'} Y_{t,\Delta,J} \mid H_{t}, A_{t,J} = 0, A_{t,J^\prime} = 0 \right] \mid S_t=s \bigg].
  \end{align*}
\noindent where~$W_{t,\Delta,j,j'} = \prod_{u=t+1}^{t+\Delta-1} \pi_u(A_{u,j}, A_{u,j'} | H_u) / p_u(A_{u,j}, A_{u,j'} | H_u)$.
\end{lemma}
\noindent Proof of Lemma~\ref{lemma:cond_effect} can be found in the Appendix~\ref{app:techdetails}.


\section{Estimation}
\label{section:estimation}

% Motivated by the identification result in Lemma~\ref{lemma:cond_effect}, we consider estimation of direct and indirect effects along with uncertainty assessments.

\subsection{Direct Causal Excursion Effect Estimation}

\begin{assumption} \normalfont
\label{ass:directeffect}
Assume the direct causal excursion effect~$\beta_{{\bf p}, \pi, \Delta}(t;s) = f_t (s)^\top \beta^\star$ where $f_t (s) \in \mathbb{R}^q$ is a $q$-dimensional feature vector that is a function of moderator state $s$ and decision point $t$. \end{assumption}

Consider inference on the $q$-dimensional parameter $\beta^\star$. Define the weight~$W_{t,j}$ at decision time $t$ for the $j$th individual as equal to $\frac{\tilde p_t (A_{t,j} \mid S_t)}{p_t (A_{t,j} \mid H_t)}$ where $\tilde p_t (a \mid S_t)\in (0,1)$ is arbitrary as long as it does not depend on terms in $H_t$ other than $S_t$, and $p(A_{t,j} \mid H_t)$ is the marginal probability that individual $j$ receives treatment $A_{t,j}$ given $H_t$.  
% For now, we consider pre-specified $\tilde p_t (a \mid S_t)$  (i.e., does not depend on the observed MRT data). 
Here we consider an estimator which is the minimizer of a cluster-based, weighted-centered least-squares (C-WCLS) criterion:
\begin{equation}
\label{eq:directwcls}
\mathbb{P}_M \left[ \frac{1}{G_m} \sum_{j=1}^{G_m} \sum_{t=1}^T W_{t,j} \times W_{t,\Delta,j} \left(Y_{t,\Delta,j} - g_t (H_t)^\top \alpha - (A_{t,j} - \tilde p_t (A_{t,j} \mid S_t)) f_t(S_t)^\top \beta \right)^2 \right]
\end{equation}
where $\mathbb{P}_M$ is defined as the average of a function over the sample, which in this context is the sample of \emph{clusters} rather than the sample of \emph{individuals} as in traditional MRT settings. In Appendix~\ref{app:techdetails}, we prove the following result.

\begin{lemma}
\label{lemma:asymnorm}
Under assumption~\ref{ass:directeffect}, given invertibility and moment conditions, the estimator $\hat \beta$ that minimizes \eqref{eq:directwcls} satisfies $\sqrt{M} \left(\hat \beta - \beta^\star \right) \to N(0, Q^{-1} W Q^{-1})$ where
$$
Q = \mathbb{E} \left[ \sum_{t=1}^T \tilde p_t( 1 \mid S_{t} ) ( 1- \tilde p_t( 1 \mid S_{t} )) f_t (S_t) f_t (S_t)^\top \right]
$$
and
\begin{align*}
W =  \mathbb{E} \bigg[ \sum_{t=1}^T &W_{t,J} \times W_{t,\Delta,J} \, \epsilon_{t,J} ( A_{t,J} - \tilde p_t( 1 \mid S_{t} )) f_t (S_t) \\
\times \sum_{t=1}^T &W_{t, \tilde J} \times W_{t,\Delta,\tilde J} \, \epsilon_{t, \tilde J} ( A_{t, \tilde J} - \tilde p_t( 1 \mid S_{t} )) f_t (S_t)^\top  \bigg]
\end{align*}
where $\epsilon_{t,j} = Y_{t,\Delta, j} - g_t(H_t)^\top \alpha^\star - (A_{t,j} - \tilde p_t (1 \mid S_t) ) f_t (S_t)^\top \beta^\star$, $\alpha^\star$ minimizes the least-squares criterion $\mathbb{E}  \left[G_m^{-1} \sum_{j=1}^{G_m} \sum_{t=1}^T W_{t,j} W_{t,\Delta,j} \left( Y_{t,\Delta,j} - g_t(H_t)^\top \alpha \right)^2 \right]$, and both $J$ and $\tilde J$ are independent randomly sampled indices from the same cluster.
\end{lemma}

\noindent In practice, plug-in estimates $\hat Q$ and $\hat W$ are used to estimate the covariance structure; Appendix~\ref{app:ssa} presents their estimates with small-sample adjustments.

% It is clear from Lemma~\ref{lemma:asymnorm} that standard MRT data analytic methods, i.e., equation~\eqref{eq:mrtstandard}, will produce biased estimates for the marginal effect of interest if cluster size is variable.

\begin{remark}{\it ($L_2$ Projection Interpretation)}
\label{rmk:l2proj}
Importantly, Assumption~\ref{ass:directeffect} is not required.  That is, we can follow~\cite{Neugebauer2007,Rosenblum2010,Kennedy2019,DempseyAOAS} and others in using $f_t (s)^\top \beta$ as a working model for $\beta_{{\bf p}, \pi, \Delta} (t;s)$.  Specifically,~$\hat \beta$ is a solution to the weighted least-squares projection
$$
\beta^\star = \arg \min_{\beta} \mathbb{E} \left[ \frac{1}{G} \sum_{j=1}^G \sum_{t=1}^T \tilde p_t(1 \mid S_t) \left( 1 - \tilde p_t(1 \mid S_t) \right) \left( \beta(t; S_t) - f_t (S_t)^\top \beta \right)^2 \right].
$$
Here, the weight is the variance of the numerator in the weight~$W_{t,j}$.  To interpret as a projection or as a correctly specified causal effect can be viewed as a bias-variance trade-off. The projection interpretation guarantees well-defined parameter interpretation in practice where Assumption~\ref{ass:directeffect} is unlikely to hold.  See~\citet[Sec. 3.1, pp.9--10][]{Kennedy2019} for a discussion of the use of projections in causal versus predictive settings.
\end{remark}

Appendix \ref{sec:semipar} presents semiparametric efficiency theory in the special case of $S_t = H_t$ that motivates the C-WCLS approach . 
% Appendix \ref{section:samesies} presents conditions that guarantee the equivalence of C-WCLS to the standard WCLS approach in terms of effect estimates and asymptotic variance.

\subsection{Connection to the Standard MRT Analysis}
\label{section:samesies}
A natural question is whether there are conditions such that the standard MRT analysis presented in Section~\ref{section:standardmrtmethods} is equivalent to the proposed direct effect analysis.  
Lemma~\ref{lemma:samesies} proves that, under certain conditions, an equivalence of estimates and asymptotic variances is guaranteed.

\begin{lemma}
\label{lemma:samesies}
Consider the direct effect when the moderator is defined on the individual (i.e., $S_{t,j}$), and the randomization probabilities only depend on the individual's observed history, i.e., $p(A_{t,j} \mid H_t) = p(A_{t,j} \mid H_{t,j})$.  If cluster size is constant (i.e., $G_m \equiv G$), then the point estimates from \eqref{eq:mrtstandard} and~\eqref{eq:directavglineareffect} are equal for any sample size. Moreover, if
\begin{equation}
    \label{eq:samesiescondition}
    \E \left[ \E \left[ W_{t,\Delta,j} \epsilon_{t,j} \times W_{t^\prime, \Delta, j^\prime} \epsilon_{t^\prime, j^\prime} \given H_{t,j}, A_{t,j}=a, H_{t^\prime,j^\prime}, A_{t^\prime,j^\prime} = a^\prime \right] \mid S_{t,j}, S_{t^\prime,j^\prime} \right]
\end{equation}
equals $\psi(S_{t,j}, S_{t^\prime,j^\prime})$
for some function $\psi$, i.e., the cross-terms are constant in $a$ and $a^\prime$, where $\epsilon_{t,j}$ is the error defined in Lemma~\ref{lemma:asymnorm}, then the estimators share the same asymptotic variance.
\end{lemma}

Proof of Lemma~\ref{lemma:samesies} can be found in Appendix~\ref{app:samesies}. Here, a class of random effect models is introduced to help with interpretation of the sufficient condition~\eqref{eq:samesiescondition}.  Specifically, for participant $j$ at decision time~$t$, let $\Delta =1$ and suppose the generative model for the proximal response is
$$
Y_{t,1,j} = g_t(H_{t,j})^\top \alpha + \underbrace{Z_{t,j}^\top b_g}_{(I)} + (A_{t,j} - p_t(1 \mid H_{t,j})) (f_t( H_{t,j})^\top \beta + \underbrace{Z_{t,j}^\top \tilde b_g}_{(II)}) + e_{t,j}
$$
where $(I)$ and $(II)$ are random effects with \textbf{}design matrix $Z_{t,j}$, $\E[ f_t(H_{t,j})^\top \beta \mid S_{t,j} ] = f_t(S_{t,j})^\top \beta$, and $e_{t,j}$ is a participant-specific error term.  The treatment effect conditional on the complete observed history and the random effects is $f_t( H_{t,j})^\top \beta + Z_{t,j}^\top \tilde b_g$, which implies  the marginal causal effect is $f_t (S_{t,j})^\top \beta$ so Assumption~\ref{ass:directeffect} holds.  Random effects in $(I)$ allow for cluster-level variation in baseline values of the proximal response, while random effects in $(II)$ allow for cluster-level variation in the fully-conditional treatment effect. Given the above generative model, sufficient condition \eqref{eq:samesiescondition} holds if $\tilde b_g \equiv 0$, i.e., when the treatment effect does not exhibit cluster-level variation.
% To see this, note that the errors $\epsilon_{t,j}$ and $\epsilon_{t,j^\prime}$ as defined in Lemma~\ref{lemma:asymnorm} do not depend on treatment if $\tilde b = 0$.  The cross-term is non-zero ($\psi \not \equiv 0$) because there is within-cluster variation due to random effects in $(I)$.  
For this reason, \eqref{eq:samesiescondition} is referred to as a treatment-effect heterogeneity condition.
% i.e., when clusters exhibit treatment-effect heterogeneity this induces marginal residual correlation to depend on $a$ and $a^\prime$ which means the proposed approach is necessary for assessing direct effects rather than standard MRT analyses. 
The condition motivates our simulation study in Section~\ref{section:sims}, which empirically supports this conclusion.


\subsection{Pairwise Indirect Causal Excursion Effect Estimation}
\label{section:indirect}

\begin{assumption} \normalfont
\label{ass:indirecteffect}
Assume the pairwise indirect causal excursion effect~$\beta^{(IE)}_{{\bf p}, \pi, \Delta}(t;s) = f_t (s)^\top \beta^{\star \star}$, where $f_t (s) \in \mathbb{R}^q$ is a $q$-dimensional vector function of $s$ and time $t$.
\end{assumption}

Consider inference on the $q$-dimensional parameter $\beta^{\star \star}$. Define the weight~$W_{t,j, j^\prime}$ at decision time $t$ for the $j$th individual as equal to $\frac{\tilde p (A_{t,j}, A_{t,j^\prime} \mid S_t)}{p_t (A_{t,j}, A_{t,j^\prime} \mid H_t)}$ where $\tilde p_t (a, a^\prime \mid S_t)\in (0,1)$ is arbitrary as long as it does not depend on terms in $H_t$ other than $S_t$, and $p(A_{t,j}, A_{t,j^\prime} \mid H_t)$ is the marginal probability that individuals $j$ and $j^\prime$ receive treatments $A_{t,j}$ and $A_{t,j^\prime}$ respectively given $H_t$.  Here we consider an estimator which is the minimizer of the following cluster-based weighted-centered least-squares (C-WCLS) criterion:
\begin{align}
\mathbb{P}_M \bigg[ \frac{1}{G_m (G_m-1)} \sum_{j \neq j^\prime} \sum_{t=1}^T &W_{t,j, j^\prime} \times W_{t,\Delta, j,j^\prime} \times\bigg(Y_{t,\Delta,j} - \nonumber\\
&g_t (H_t)^\top \alpha - (1-A_{t,j}) (A_{t,j^\prime} - \tilde p_t^\star (1\mid S_t)) f_t(S_t)^\top \beta \bigg)^2 \bigg ] \label{eq:indirectwcls}
\end{align}
where $\tilde p_t^\star (1\mid S_t) = \frac{\tilde p_t (0,1 \mid S_t)}{\tilde p_t (0,0 \mid S_t) + \tilde p_t (0,1 \mid S_t)}$ and $W_{t,\Delta,j,j^\prime} = \prod_{u=t+1}^{t+\Delta-1} \pi(A_{u,j}, A_{u,j^\prime} | H_u) / p(A_{u,j}, A_{u,j^\prime} | H_u)$.  If an individual's randomization probabilities only depends on their own observed history then $\tilde p_t^\star (1 \mid S_{t,j^\prime}) = \tilde p_t (1 \mid S_{t,j^\prime})$ and the weight~$W_{t,\Delta, j,j^\prime} = W_{t,\Delta,j} \times W_{t, \Delta,j^\prime}$.  In Appendix~\ref{app:techdetails}, we prove the following result.

\begin{lemma}
\label{lemma:asymnorm2}
Under assumption~\ref{ass:indirecteffect}, then, under invertibility and moment conditions, the estimator $\hat \beta^{(IE)}$ that minimizes \eqref{eq:indirectwcls} satisfies $\sqrt{M} \left( \hat \beta^{(IE)} - \beta^{\star \star} \right) \to N(0, Q^{-1} W Q^{-1})$ where
$$
Q = \mathbb{E} \left[ \sum_{t=1}^T (\tilde p_t (0,0 \mid S_t) + \tilde p_t (0,1 \mid S_t)) \tilde p_t^\star ( 1 \mid S_{t} ) ( 1- \tilde p_t^\star ( 1 \mid S_{t} )) f_t (S_t) f_t (S_t)^\top \right]
$$
and
\begin{align*}
W =  &\mathbb{E} \bigg[ \sum_{t=1}^T W_{t,J,J^\prime} W_{t,\Delta, J,J^\prime} \epsilon_{t,J,J^\prime} (1-A_{t,J})( A_{t,J^\prime} - \tilde p_t^\star( 1 \mid S_{t} )) f_t (S_t) \\
&\times \sum_{t=1}^T W_{t,\tilde J,\tilde J^\prime} \times W_{t, \Delta, \tilde J, \tilde J^\prime} \epsilon_{t, \tilde J, \tilde J^\prime} (1-A_{t,\tilde J}) ( A_{t,\tilde J^\prime} - \tilde p_t^\star( 1 \mid S_{t} )) f_t (S_t)^\top  \bigg]
\end{align*}
where $\epsilon_{t,j,j^\prime} = Y_{t,\Delta,j} - g_t(H_t)^\top \alpha^{\star \star} - (1-A_{t,j}) (A_{t,j^\prime} - \tilde p_t^\star (1 \mid S_t) ) f_t (S_t)^\top \beta^{\star \star}$,$\alpha^{\star \star}$ minimizes the least-squares criterion $\mathbb{E}  \left[ \frac{1}{G_m (G_m-1)} \sum_{j \neq j^\prime} \sum_{t=1}^T W_{t,j,j^\prime} W_{t,\Delta, j,j'} \left( Y_{t,\Delta,j} - g_t(H_t)^\top \alpha \right)^2 \right]$, and both $(J,J^\prime)$ and $(\tilde J, \tilde J^\prime)$ are independently, randomly sampled pairs from the cluster.
\end{lemma}

\begin{remark}%[Data-driven numerator]
In Appendix \ref{app:addindirect}, variances are re-derived in the general setting where the numerators $\tilde p_t (a \mid S_t)$ and $\tilde p_t (a, a^\prime \mid S_t)$ are estimated using the observed MRT data.  
\end{remark}



\section{Simulations}
\label{section:sims}

To evaluate the proposed estimator, we extend the simulation setup in \cite{Boruvkaetal}. We first present a base data-generation model, which is to be extended in four scenarios. Consider an MRT with known randomization probability and the observation vector $O_t$ is a single state variable $S_t \in \{-1,1\}$ at each decision time $t$. We focus on lag-1 proximal responses ($\Delta = 1$):
\begin{equation}
\label{eq:generativemodel}
\begin{array}{r@{}l}
    Y_{t,1} = &{} \theta_1 \{S_t - \E \left[ S_t|A_{t-1},H_{t-1}\right]\} + \theta_2\{A_{t-1}-p_{t-1}(1|H_{t-1})\} \\
    &{}+ \{A_t - p_t(1|H_t)\}(\beta_{10} + \beta_{11} S_t)+  e_{t+1}.
\end{array}
\end{equation}
The randomization probability is $p_t(1|H_t) = \text{expit}(\eta_1 A_{t-1}+\eta_2 S_t)$ where $\text{expit}(x)=(1+\exp(-x))^{-1}$; the state dynamics are given by $\mathbb{P}(S_t=1|A_{t-1},H_{t-1})=\text{expit}(\xi A_{t-1})$ with $A_0 = 0$, and the independent error term satisfies $e_t \sim \mathcal{N}(0,1)$ with $\text{Corr}(e_u, e_t) = 0.5^{|u-t|/2}$. As in~\cite{Boruvkaetal}, we set $\theta_1=0.8, \theta_2 = 0, \xi=0, \eta_1 = -0.8, \eta_2 = 0.8, \beta_{10}=-0.2$, and $\beta_{11} = 0.2$. Because $\xi=0$, the marginal proximal effect is equal to $\beta_{10} + \beta_{11} \E \left[S_t \right]=\beta_{10} = -0.2$. 
% The marginal treatment effect is thus constant in time and is given by $\beta_1^\star = \beta_{10} =  -0.2$.
%We consider four simulation scenarios based on the above generative model. The first three concern estimation of the marginal proximal treatment effect. We set $f_t(S_t)=1$ in~\eqref{eq:generativemodel} (i.e., $S_{1t} = \emptyset$) and report average point estimate, standard deviation (SE), root mean squared error (RMSE), and 95\% confidence interval coverage probabilities (CP) across 1000 replicates. We vary the number of clusters and cluster size.  We compare the proposed cluster-based method (C-WCLS) to the approach by~\cite{Boruvkaetal} (WCLS).
In extending the data generation model to settings with clustered settings, we conducted simulation studies with $25, 50, 100$ clusters with equal sizes ($15$ or $25$); The Main Paper reports results with $50$ clusters showing the relative advantage of C-WCLS over WCLS. More complete simulation results with similar findings can be found in Appendix~\ref{app:simdetails}.


\begin{figure}
  \centering
  \figuresize{0.8}
  \figurebox{15pc}{20pc}{}[coveragecomparison.eps]
  \caption{C-WCLS offers valid $95\%$ confidence intervals in Scenario II. WCLS does not. Empirical coverage varies by group size  ($G$) ({\bf left}), and relative variance of $b_g$ as compared to $e_g$ (\bf right).}
  \label{fig:undercoverage}
\end{figure}

% \hs{1. The 3 sims $\to$ show with no $b_g$ we recover same coverage? \zw{group-specific $b_g$ interacting or not interacting with the centered treatment. show valid coverage when $b_g$ does not interact with the centered treatments.}}

\noindent {\bf Simulation Scenario I}. The first scenario estimates the marginal proximal effect when an individual-level moderator exists and proximal responses share a random cluster-level intercept term that does not interact with treatment. The data generative model~\eqref{eq:generativemodel} incorporates a cluster-level random-intercept $e_g \sim \mathcal{N}(0,0.5)$.
% , so that $
% Y_{t+1,j} = (-0.2 + 0.2 \cdot S_{t,j}) \times (A_{t,j} -p_t(1|H_{t,j})) + 0.8 S_{t,j} + e_g +e_{t+1,j}$.
Table~\ref{tab:simresults} presents the results, which shows both WCLS and the proposed C-WCLS approach achieve nearly unbiasedness and proper coverage. This is in line with Lemma~\ref{lemma:samesies} stating asymptotic equivalence under no cluster-level treatment heterogeneity.
% This demonstrates that the performance of the WCLS approach is not impacted by group-level correlation that does not interact with treatment.

\noindent {\bf Simulation Scenario II}. In the second scenario, we extend Scenario I to include a random cluster-level intercept term that interacts with treatment by considering the linear model with the additional term~$b_g \times (A_{t,j} -p_t(1|H_{t,j}))$
%\begin{equation}
%Y_{t+1,j} = (-0.2 + b_g +  0.2 \cdot S_{t,j}) \times (A_{t,j} -p_t(1|H_{t,j})) + 0.8 \cdot S_{t,j} + e_g + e_{t+1,j}
%\end{equation}
where $b_g \sim \mathcal{N}(0,0.1)$.
% is a random-intercept term within the treatment effect per cluster. 
Table~\ref{tab:simresults} presents the results which demonstrate that if cluster-level random effects interact with treatment, then both methods produce nearly unbiased estimates of the marginal proximal effect but only the proposed method achieves the nominal 95\% coverage probability. To further demonstrate this, Figure~\ref{fig:undercoverage} presents nominal coverage as a function of the ratio of the variance of $b_g$ over the variance of~$e_g$ as well as group size respectively.  
% This shows that the coverage probability of the WCLS method decays rapidly while the proposed method achieves the nominal 95\% coverage probability for all choices of the variance of $b_g$ and when the group size increases.  
% In Figure~\ref{fig:undercoverage} (left panel), note that even when group size is $5$ (i.e., small groups), the nominal coverage drops to 80\%.

\noindent {\bf Simulation Scenario III}.  In the third scenario, the treatment effect for an individual is assumed to depend on the average state of all individuals in the cluster, i.e., define the cluster-level moderator $\bar S_{t,g} = \frac{1}{G_g}\sum_{j=1}^{G_g} S_{t,j}$ and consider the linear model from Scenario II with the additional term~$\bar S_{t,g} \times (A_{t,j} -p_t(1|H_{t,j}))$.
%\begin{equation*}
% Y_{t+1,j} = (-0.2 + b_g +  0.2 \cdot \bar S_{t,g}) \times (A_{t,j} -p_t(1|H_{t,j})) + 0.8 S_{t,j} + e_g + e_{t+1,j}
% \end{equation*}
The proposed estimator again achieves the nominal 95\% coverage probability while the WCLS method does not (see Scenario III, Table~\ref{tab:simresults}).


\begin{table}[!th]
\def~{\hphantom{0}}
\tbl{\it Simulation: C-WCLS and WCLS comparison for Scenarios I, II, III, and IV.}{%
\begin{tabular}{c ccccccc}
\\
% \hline
Scenario & Estimator & \# of Clusters & Cluster Size & Estimate & SE & RMSE & CP \\ %\hline
\multirow{4}{*}{I} 
% & C-WCLS & \multirow{2}{*}{25} & \multirow{2}{*}{10} & -0.198 & 0.035 & 0.036	 & 0.945 \\
% & WCLS & & &  -0.200 & 0.036 & 0.035 & 0.956 \\  \cdashline{2-8}
% & C-WCLS & \multirow{2}{*}{25} & \multirow{2}{*}{25} & -0.199 & 0.022 & 0.023 & 0.948 \\
% & WCLS & & &  -0.199 & 0.023 & 0.022 & 0.958 \\ \cdashline{2-8}
& C-WCLS & \multirow{2}{*}{50} & \multirow{2}{*}{10} & -0.198 & 0.025 & 0.027 & 0.935 \\
& WCLS & & &  -0.198 & 0.026 & 0.026 & 0.944 \\ %\cdashline{2-8}
& C-WCLS & \multirow{2}{*}{50} & \multirow{2}{*}{25} & -0.198 & 0.016 & 0.016 & 0.950 \\
& WCLS & & &  -0.198 & 0.016 & 0.017 & 0.937 \\ %\hline  %\cdashline{2-8}
&  & & &   &  &  &  \\
% & C-WCLS & \multirow{2}{*}{100} & \multirow{2}{*}{10} & -0.199 & 0.018 & 0.018 & 0.949 \\
% & WCLS & & &  	-0.198& 0.018 & 0.019 & 0.949 \\ \cdashline{2-8}
% & C-WCLS & \multirow{2}{*}{100} & \multirow{2}{*}{25} &  -0.198 & 0.011 & 0.012  & 0.941 \\
% & WCLS & & &  -0.199 & 0.011 & 	0.012 & 0.944 \\ \hline
\multirow{4}{*}{II} 
% & C-WCLS & \multirow{2}{*}{25} & \multirow{2}{*}{10} & -0.199 & 0.070 & 0.076 & 0.935 \\
% & WCLS & & &  -0.201 & 0.041 & 0.076 & 0.710 \\  \cdashline{2-8}
% & C-WCLS & \multirow{2}{*}{25} & \multirow{2}{*}{25} & -0.196 & 0.065 & 0.071 & 	0.933 \\
% & WCLS & & &  -0.200 & 0.026 & 0.065 & 0.557 \\ \cdashline{2-8}
& C-WCLS & \multirow{2}{*}{50} & \multirow{2}{*}{10} & -0.200 & 0.051 & 0.049 & 0.957 \\
& WCLS & & &  -0.200 & 0.029 & 0.052 & 0.723 \\ %\cdashline{2-8}
& C-WCLS & \multirow{2}{*}{50} & \multirow{2}{*}{25} & -0.200 & 0.047 & 0.049 & 0.947 \\
& WCLS & & &  -0.199 & 0.019 & 0.048 & 0.555 \\ %\hline %\cdashline{2-8}
&  & & &   &  &  &  \\
% & C-WCLS & \multirow{2}{*}{100} & \multirow{2}{*}{10} & -0.198  & 0.036 & 0.035 & 0.955 \\
% & WCLS & & &  -0.198 & 0.021 & 0.037 & 0.718 \\ \cdashline{2-8}
% & C-WCLS & \multirow{2}{*}{100} & \multirow{2}{*}{25} & -0.198  & 0.033 & 0.035 & 0.942 \\
%& WCLS & & &  -0.199 & 0.013 & 0.032 & 0.583 \\ \hline
\multirow{4}{*}{III} 
% & C-WCLS & \multirow{2}{*}{25} & \multirow{2}{*}{10} & -0.199 & 0.070 & 0.073 & 0.948 \\
% & WCLS & & &  -0.202 & 0.041 & 0.074 & 0.734 \\  \cdashline{2-8}
% & C-WCLS & \multirow{2}{*}{25} & \multirow{2}{*}{25} & -0.196 & 0.066 & 0.069 & 0.931 \\
% & WCLS & & &  -0.200 & 0.026 & 0.068 & 0.563 \\ \cdashline{2-8}
& C-WCLS & \multirow{2}{*}{50} & \multirow{2}{*}{10} & -0.198 & 0.051 & 0.052 & 0.941 \\
& WCLS & & &  -0.199 & 	0.029 & 0.052 & 0.742 \\ %\cdashline{2-8}
& C-WCLS & \multirow{2}{*}{50} & \multirow{2}{*}{25} & -0.199 & 0.047 & 0.048 & 0.946 \\
& WCLS & & &  -0.200 & 0.018 & 0.048 & 0.561 \\% \hline %\cdashline{2-8}
&  & & &   &  &  &  \\
% & C-WCLS & \multirow{2}{*}{100} & \multirow{2}{*}{10} & -0.200 & 0.036 & 0.037 & 0.946 \\
% & WCLS & & &  	-0.200 & 0.021 & 0.037 & 0.740 \\ \cdashline{2-8}
% & C-WCLS & \multirow{2}{*}{100} & \multirow{2}{*}{25} & -0.201 & 0.034 & 0.033 & 0.956 \\
% & WCLS & & &  -0.198 & 0.013 & 0.034 & 0.555 \\ \hline
\multirow{2}{*}{IV}
& \multirow{2}{*}{C-WCLS} & \multirow{2}{*}{50} & 
10 & -0.097 & 0.020 & 0.021 & 0.953 \\
& & & 25 & -0.100 & 0.013 & 0.013 & 0.942 \\ %\hline
\end{tabular}}
\label{tab:simresults}
%\begin{tabnote}
%U.S., United States of America; R, respondent.
%\end{tabnote}
\end{table}


% \hs{4. Demonstrate indirect effect}

\noindent {\bf Simulation Scenario IV}.  The fourth scenario considers the indirect effect.  For individual $j$ at decision point $t$, define the total effect to be $TE_{t,j} = \sum_{j^\prime \neq j} \{A_{t,j^\prime} - \tilde p_{t, j^\prime} ( 1 \mid H_t) \} (\beta_{20} + \beta_{21} S_{t,j^\prime})$, where $\beta_{20} = -0.1$ and $\beta_{21} = 0.2$. The generative model is then given by:
\begin{equation*}
    Y_{t,1,j} = (-0.2 + b_g +  0.2 \cdot \bar S_{t,g}) \times \{A_{t,j} -p_t(1|H_{t,j})\}+ 0.8 S_{t,j} +TE_{t,j} +e_g +e_{t+1,j}
\end{equation*}
This model implies a marginal pairwise indirect effect equal to $\beta^{(IE)} = \beta_{20} = -0.1$. Table~\ref{tab:simresults} presents simulation results which shows that the proposed indirect estimator exhibited nearly no bias and achieved the nominal coverage probability. 

\noindinet \begin{remark}
A simulation study of direct and pairwise indirect effects with $\Delta>1$ is presented in Appendix \ref{app:lagsimulation}.
\end{remark} 
% \begin{table}[!th]
% \def~{\hphantom{0}}
% \tbl{\it Simulations show strong finite sample estimation and accurate coverage for indirect effects.}{%
% \begin{tabular}{c | cccccc}
% \hline
% Scenario & \# of Clusters & Cluster Size & Estimate & SD & RMSE & CP \\ \hline
% \multirow{6}{*}{IV} & 25 & 10 & -0.097 & 0.028 & 0.029 & 0.958 \\
% & 25 & 25 & -0.101 & 0.017 &  0.019 & 0.942 \\
% & 50 & 10 & -0.097 & 0.020 & 0.021 & 0.953 \\
% & 50 & 25 & -0.100 & 0.013 & 0.013 & 0.942 \\
% & 100 & 10 & -0.097 & 0.015 & 0.015 & 0.943 \\
% & 100 & 25 &  -0.100 & 0.009 & 0.009 & 0.944 \\ \hline
% \end{tabular}}
% \label{tab:simresults_indirect}
% \end{table}

\section{Case Study: Intern Health Study}
\label{section:casestudy}

The Intern Health Study (IHS) was a 6-month MRT on 1,562 medical interns where four types of weekly notification - mood, activity, sleep, or none -- were randomly assigned with equal probability to each subject~\citep{Necamp2020}; see Section~\ref{section:motex} for prior discussion.
In IHS,  285 institutions and 24 specialties were observed.   Here, we assess the effect of the three types of notifications (mood, activity, and sleep) compared to no notifications on the weekly average of self-reported mood scores, log step-count and log sleep minutes for the population of interns.  
% See Appendix~\ref{app:IHSadditionalanalysis} for an additional analysis of log sleep minutes.  
Due to high levels of missing data, weekly proximal responses were multiply imputed.  
% Similar imputation was performed for daily log step-counts and daily log sleep minutes. 
See~\cite{Necamp2020} for further details.

\begin{figure}[!th]
  \label{fig:wcls_moderation_IHS}
  \figuresize{0.8}
  \figurebox{15pc}{20pc}{}[mainpaper.eps]
  \caption{Moderation of average previous week's proximal responses on the effect of notifications on average weekly mood scores, log step counts, and log sleep counts respectively in IHS. }
\end{figure}

Let $t=1,\ldots,T$ denote the weekly decision points at which the individual is randomized to the various types of notifications.  
% Because of the form of the intervention, all participants were available for this intervention throughout the study; i.e., $I_t \equiv 1$. 
The three proximal responses are the average weekly mood score,  which is reported on a Likert scale  taking values from 1 to 10 (higher scores mean better mood), log step count and log sleep minutes respectively.
Notifications are collapsed to a binary variable, i.e., $A_{t,j}=1$ if the individual was assigned to receive any notifications on week $t$; otherwise, $A_{t,j}= 0$.
% At any occasion $t$, an individual's notification randomization probabilities were only dependent of their past observed history~$H_{t,j}$. 
For simplicity, we start with clusters constructed based on medical specialty.  The average cluster size was 65; the first and third quartile were 7 and 113 respectively, with maximum and minimum sizes of 333 and 1.  For every individual in each cluster at each decision point we compute the average prior weekly proximal response for all others in the cluster, denoted $\bar Y_{t, -j}$ for the $j$th individual in the cluster.  We conducted analyses under lag $\Delta=1$ and $\Delta=2$. We report results under $\Delta=1$; See Appendix~\ref{app:IHSadditionalanalysis} for results under $\Delta=2$ for two choices of reference policy $\pi$. Under $\Delta=1$, we consider two moderation analyses that can both be expressed as
$\beta(t; S_t) = \beta_0 + \beta_1 \cdot Y_{t,j} + \beta_2 \bar Y_{t,-j}$.

The first set of moderation analyses considers the standard moderation analysis where only individual-level moderators are included (i.e., $\beta_2 = 0$). Figure~\ref{fig:wcls_moderation_IHS} visualizes the estimates across the range of prior week's proximal response for both our proposed approach and the WCLS approach from~\cite{Boruvkaetal}, and the numerical output can be found in Appendix ~\ref{app:moreonFigure3}. In comparison, C-WCLS produces larger variance estimations for all proximal responses as expected. The effects do not change too much for the average weekly mood and sleep analysis; however, the significant effect of messages on weekly log step count under the traditional MRT analysis becomes insignificant when accounting for cluster effects.


The second moderation analysis lets $\beta_2$ be a free parameter, enabling novel moderation analyses that accounts for the average weekly previous proximal responses of other individuals.  Table~\ref{tab:IHS_direct} presents the results.  Here, we see that the new term $\beta_2$ is negative but insignificant.  The results suggest the average proximal responses of others in the cluster have a limited moderation effect. To conclude, the impact of a notification on mood is larger while the individual's score from previous week is low.   Similar results hold for the log step-count analysis.
% the constant term~$\beta_0$  while



% \begin{table}[!th]
% \def~{\hphantom{0}}
% \tbl{\it Moderation analysis with cluster-level moderators.}{%
% \begin{tabular}{c |c | crrrr}
% \hline
% Outcome & Setting & Variables & Estimate & Std. Error & t-value & p-value \\ \hline
% \multirow{3}{*}{Mood} 
% % & \multirow{2}{*}{WCLS} & $\beta_0$ & 0.369 & 0.086 & 4.268 & 0.000 \\
% % & & $\beta_1$ & -0.055 & 0.011 & -4.822 & 0.000 \\ \cline{2-7}
% % & \multirow{2}{*}{C-WCLS} & $\beta_0$ & 0.350 & 0.103 & 3.401 & 0.001 \\
% % & & $\beta_1$ & -0.053 & 0.014 & -3.868 & 0.000 \\ \cline{2-7}
% & \multirow{3}{*}{C-WCLS} & $\beta_0$ & -0.238 & 0.282 & -0.842 & 0.401 \\
% & & $\beta_1$ & -0.054 & 0.014 & -3.973 & 0.000 \\
% & & $\beta_2$ & 0.083 & 0.037 & 2.241 & 0.026 \\ \hline
% \multirow{3}{*}{Steps} 
% % & \multirow{2}{*}{WCLS} & $\beta_0$ & 0.729 & 0.295 & 2.472 & 0.015 \\
% % & & $\beta_1$ & -0.037 & 0.015 & -2.484 & 0.015 \\ \cline{2-7}
% % & \multirow{2}{*}{C-WCLS} & $\beta_0$ & 0.622 & 0.384 & 1.618 & 0.108 \\
% % & & $\beta_1$ & -0.031 & 0.019 & -1.580 & 0.117 \\ \cline{2-7}
% & \multirow{3}{*}{C-WCLS} & $\beta_0$ & -2.095 & 1.248 & -1.678 & 0.094 \\
% & & $\beta_1$ & -0.034 & 0.019 & -1.745 & 0.083 \\
% & & $\beta_2$ & 0.143 & 0.062 &2.330 & 0.020 \\ \hline
% \multirow{3}{*}{Sleep}
% & \multirow{3}{*}{C-WCLS} & $\beta_0$ & -1.912 & 1.379 & -1.386 & 0.171  \\
% & &$\beta_1$ & -0.067 & 0.023 & -2.948 & 0.004 \\
% & &$\beta_2$ & 0.162 & 0.065 & 2.487 & 0.015 \\ \hline
% \end{tabular}}
% \label{tab:IHS_direct}
% \end{table}

\begin{table}
\def~{\hphantom{0}}
\tbl{Moderation analysis using C-WCLS in IHS (lag $\Delta=1$)}{%
\begin{tabular}{lcrrrcrrr}
\\
& \multicolumn{4}{c}{Direct Effect} & \multicolumn{4}{c}{Indirect Effect} \\
& Variables & Estimate & Std. Error & p-value & Variables & Estimate & Std. Error & p-value \\[5pt]
\multirow{3}{*}{Mood} 
  & Intercept ($\beta_0$) & 0.563 & 0.251  & 0.028  & $\tilde \beta_0$ & -0.054 & 0.045  & 0.883 \\
 & Prior Week Avg. ($\beta_1$) & -0.066  &   0.027    & 0.016 & $\tilde \beta_1$ & -0.015 & 0.031  & 0.684 \\
 & Cluster Pr. Wk. Avg. ($\beta_2$) & -0.016  &   0.017  &   0.349 & & & & \\ 
%  &&  & &   & & & & &  \\
\multirow{3}{*}{Steps} 
 & Intercept ($\beta_0$) & 1.165  & 0.782   &  0.139 & $\tilde \beta_0$ & -0.038 & 0.134  & 0.612 \\
 & Prior Week Avg. ($\beta_1$) & -0.048  & 0.036 &  0.177 & $\tilde \beta_1$ & 0.019 & 0.090  & 0.417\\
 & Cluster Pr. Wk. Avg. ($\beta_2$) & -0.010  &   0.014    & 0.482 & & & & \\ %\hline
% &&  & &   &  & & & & \\
\multirow{3}{*}{Sleep}
 & Intercept ($\beta_0$) & 1.545 &  0.779  &  0.050 & $\tilde \beta_0$ & 0.007 & 0.096 & 0.469 \\
 & Prior Week Avg. ($\beta_1$) & -0.081   &  0.039  &    0.037 & $\tilde \beta_1$ & -0.004 & 0.065 &  0.526 \\
 & Cluster Pr. Wk. Avg. ($\beta_2$) & 0.000 & 0.006   & 0.961 & & & &\\
\end{tabular}}
\label{tab:IHS_direct}
\begin{tabnote}
Moderation analysis for the direct and indirect effect of notifications on average weekly mood scores, log step counts, and log sleep minutes respectively in IHS.  Coefficient $\tilde \beta_0$ represents the indirect effect under $A_{t,j} = 0$, while the coefficient $\tilde \beta_1$ represents the indirect effect under $A_{t,j} = 1$.
\end{tabnote}
\end{table}


% \begin{table}[!th]
% \def~{\hphantom{0}}
% \tbl{\it Moderation analysis with cluster-level moderators (lag $\Delta=1$).}{%
% \begin{tabular}{c c crrrr}
% %\hline
% Outcome &  Variables & Estimate & Std. Error & p-value \\ %\hline
% \multirow{3}{*}{Mood} 
% % & \multirow{2}{*}{WCLS} & $\beta_0$ & 0.369 & 0.086 & 4.268 & 0.000 \\
% % & & $\beta_1$ & -0.055 & 0.011 & -4.822 & 0.000 \\ \cline{2-7}
% % & \multirow{2}{*}{C-WCLS} & $\beta_0$ & 0.350 & 0.103 & 3.401 & 0.001 \\
% % & & $\beta_1$ & -0.053 & 0.014 & -3.868 & 0.000 \\ \cline{2-7}
%   & $\beta_0$ & 0.563 & 0.251  & 0.028 \\
%  & $\beta_1$ & -0.066  &   0.027    & 0.016 \\
%  & $\beta_2$ & -0.016  &   0.017  &   0.349 \\ %\hline
%  &&  & &   &  \\
% \multirow{3}{*}{Steps} 
% % & \multirow{2}{*}{WCLS} & $\beta_0$ & 0.729 & 0.295 & 2.472 & 0.015 \\
% % & & $\beta_1$ & -0.037 & 0.015 & -2.484 & 0.015 \\ \cline{2-7}
% % & \multirow{2}{*}{C-WCLS} & $\beta_0$ & 0.622 & 0.384 & 1.618 & 0.108 \\
% % & & $\beta_1$ & -0.031 & 0.019 & -1.580 & 0.117 \\ \cline{2-7}
%   & $\beta_0$ & 1.165  & 0.782   &  0.139 \\
%  & $\beta_1$ & -0.048  & 0.036 &  0.177 \\
%  & $\beta_2$ & -0.010  &   0.014    & 0.482 \\ %\hline
%  &&  & &   &  \\
% \multirow{3}{*}{Sleep}
%   & $\beta_0$ & 1.545 &  0.779  &  0.050  \\
%  &$\beta_1$ & -0.081   &  0.039  &    0.037 \\
%  &$\beta_2$ & 0.000 & 0.006   & 0.961 \\ %\hline
% \end{tabular}}
% \label{tab:IHS_direct}
% \end{table}

Finally, we consider indirect moderation effect analyses.   In this analysis, clusters are defined based on medical specialty and institution because interference was only likely when interns are in close geographic proximity.   Here, we consider the marginal indirect effect (e.g., no moderators) both when the individual did not receive the intervention and when the individual did receive an intervention at decision time~$t$.  Table~\ref{tab:IHS_direct} presents the results.  In this case, the estimated indirect effects are much weaker than the direct effects.  Even a weak effect may be unexpected as none of the content in the push notifications was aimed at impacting other individuals' behavior.  For all the proximal responses, we see limited evidence of an indirect effect.  This implies that the scientific team, when building an optimal intervention package, may ignore these indirect effects and focus solely on the individual who receives these types of push notifications.
% Existence of signal for the impact on proximal mood implies the scientific team may wish to consider interventions that target these indirect effects more explicitly, either in the framing of the messages or in messages aimed at impacting the overall cluster.
% \begin{table}[!th]
% \def~{\hphantom{0}}
% \tbl{\it Moderation analysis for the indirect effect of notifications on average weekly mood scores, log step counts, and log sleep minutes respectively in IHS.  Coefficient $\tilde \beta_0$ represents the indirect effect under $A_{t,j} = 0$, while the coefficient $\tilde \beta_1$ represents the indirect effect under $A_{t,j} = 1$.}{%
% \begin{tabular}{ccrrrr}
% %\hline
% Outcome & Variables & Estimate & Std. Error  & p-value \\ %\hline
% \multirow{2}{*}{Mood} & $\tilde \beta_0$ & -0.054 & 0.045  & 0.883 \\
% & $\tilde \beta_1$ & -0.015 & 0.031  & 0.684 \\ %\hline
% & &  & & & \\
% \multirow{2}{*}{Steps} & $\tilde \beta_0$ & -0.038 & 0.134  & 0.612\\
% & $\tilde \beta_1$ & 0.019 & 0.090  & 0.417 \\ %\hline
% & &  & & & \\
% \multirow{2}{*}{Sleep} & $\tilde \beta_0$ & 0.007 & 0.096 & 0.469\\
% & $\tilde \beta_1$ & -0.004 & 0.065 &  0.526\\ %\hline
% \end{tabular}}
% \label{tab:IHS_indirect}
% \end{table}

\section{Discussion}

Here we consider the causal excursion effect in the presence of a priori known clusters.  We have extended the causal excursion effect to naturally account for cluster information. In particular, both direct and indirect excursion effects have been formalized in the context of MRT to account for potential interference.  The effects described in this paper are most important when using MRT data to build optimized JITAIs for deployment in an mHealth package. Specifically, the estimation procedure for the direct excursion effect accounts for within-cluster correlation in the proximal responses which helps the scientific team avoid making erroneous conclusions about intervention effectiveness using standard MRT methods.  Moreover, estimation of indirect effects allows the scientific team to answer questions about impact of interventions on other members of the same cluster.  Use of these methods provides empirical evidence for the scientific team to include or exclude intervention components that may have had unanticipated second order effects, or potentially lead to novel ways to improve the intervention component by revising the intervention to more explicitly account for cluster-level interference. While this work represents a major step forward in the analysis of micro-randomized trial data, further work is required.  Specifically, extensions to be considered future work include accounting for overlapping communities and/or network (rather than cluster-only) structure~\citep{Ogburn2014,Mealli2019}, accounting for general non-continuous proximal responses such as binary or count outcomes~\citep{Qian2021}, penalization of the working model to allow for high-dimensional moderators, and a method to use the proposed approach to form warm-start policies at the individual level while accounting for group level information~\citep{Luckett2020}.


% \section*{Acknowledgments}
% The authors would like to thank the Intern Health Study (IHS) team at University of Michigan for substantive discussions, Professor Srijan Sen (Principal Investigator) for generously providing access to the IHS data, and the study participants of IHS for providing the data.  The draft also benefited from helpful comments from Professors Inbal Nahum-Shani and Peng Ding.

\bibliographystyle{biometrika}
\bibliography{paper-ref}

% \newpage

% \section{Address}

% For each author please give one postal address, including a department, postcode and country, and one e-mail address; these should be the best permanent addresses current at time of publication. Acknowledgements to other institutions should be put with other acknowledgements at the end of the paper. Names of states should be given in full, thus: California rather than CA, S\~ao Paulo rather than SP. Use U.S.A. and U.K. Note that England, Scotland and Wales should not be used.

% \section{Length}

% The average length for papers published in recent years is just under 13 sides.  The probability of acceptance drops sharply beyond this length, particularly if it is felt that a paper is long in relation to its original content.  Authors should endeavour to write as concisely as possible,  consistent with clarity.  Long or standard derivations should be omitted, referenced elsewhere, or made available in a supplementary document on the \Bka\ web site; see page~\pageref{SM}.  Essential technical details may be placed in an appendix.

% The maximum length for a paper in the Miscellanea section is 8 journal sides.

% \section{Style}

% \subsection{Sections, subsections and paragraphs}

% If subsections are used to divide a section, no text should appear before the first subsection; all text should appear within numbered subsections.  Subsubsections are not used.

% The end of a paragraph is marked in the .tex file by a blank line.  Extra characters such as \verb+\\+ at the end of lines or paragraphs should not be used.  Bad line breaks are corrected during the production process.

% \subsection{Spelling, abbreviations and special symbols}

% English spelling is used, with Oxford ``-ize'' endings.

% Verbal phrases inside brackets or dashes, or in italic or bold type, should not be used. Quotation marks should be used only for direct quotations, which should be attributed. Footnotes should be avoided except for tables.

% Abbreviations like a.s., i.i.d., d.f., w.r.t.,  ANOVA, AR, MCMC, MAP and ML may not be used.  Exceptions to this are common non-mathematical abbreviations such as DNA and HIV, which appear as ordinary upper-case letters, and, in exceptional cases, where the use of an abbreviation clearly improves the readability of the paper.

% Do not create abbreviations to describe methods. Thus `our method is more efficient than Wellner and Zhang's method' should replace `our new method is better than method WZ'.

% Special symbols like $\mathop{\rightarrow}\limits^{\rm d}$, $\forall$, $\exists$, $:=$ and $=:$  should not be used.  The symbol $\mid$ should not be used in text as shorthand, and in mathematics the \TeX\ symbol \verb+\mid+ should be used to denote conditioning, rather than \verb+|+ or \verb"\vert".

% Symbols comprising several letters such as \AIC\ or \textsc{ar}$(p)$ may be used as mathematical objects if previously defined. They may not be used as abbreviations for English words; thus `the \textsc{ar} model' should be `the autoregressive model'.   In such cases small capital letters, for example the  \TeX\ syntax \verb+\textsc{aic}+ for \AIC,\ are used; consistency is best assured by defining a macro at the start of the \texttt{.tex}  file.

% One of the most common reasons that publication of scientifically acceptable papers is delayed is authorial  failure to adhere to journal policy on abbreviations, so it may be worthwhile to explain why \Bka\ eschews them. The purpose of scientific writing is to convey ideas as clearly and directly as possible.  Abbreviations militate against this: a reader who does not know them will spend time looking back through the paper to find what they mean, and they lead to sloppy mechanistic writing. A sentence such as `MLE for a GLMM may be performed using the BFGS, NR, CG or EM algorithms, but MCMC is an alternative' forces the reader to waste energy on parsing acronyms  rather than focusing on the underlying ideas.

% \subsection{English}

% English sentences containing mathematical expressions or displayed formulae should be punctuated in the usual way: in particular please check carefully that all displayed expressions are correctly punctuated.  Displayed expressions should be preceded by a colon only if grammatically warranted. Do not place a colon in the middle of a clause.

% Words in common terms such as central limit theorem or Brownian motion are capitalized only if they are derived from proper names: thus bootstrap,  lasso and mean square error rather than Bootstrap, Lasso and  Mean Square Error.

% Hyphens - (\verb+-+ in \TeX), n-dashes -- (\verb+--+), m-dashes --- (\verb+---+), and minus signs $-$ (\verb+$-$+) have different uses.  Hyphens are used to join two words, or in the double-barrelled name of a single person (e.g.\ non-user, Barndorff-Nielsen); n-dashes are used in ranges of numbers or to join the names of two different people (1--7, Neyman--Pearson); and minus signs are used in mathematics (e.g. $-2$). m-dashes are not used in \Bka.\  Parenthetical remarks, like this subordinate clause, are placed between two commas.

% Two bugbears: the phrase `note that' can almost always be deleted, and the phrase `is given by' should be cut to `is' in a sentence such as `The average is given by $\bar X=n^{-1}(X_1+\cdots+X_n)$.

% \subsection{Mathematics}

%  Equation numbers should be included only when equations are referred to; the
% numbers must be placed on the right. Long or important mathematical, not verbal,
% expressions should be displayed, i.e., shown on a separate line. Short formulae should be
% left in the text to save space where possible, but must not be more than one line high and
% not contain reduced-size type. For example $\frac{{\rm d}y}{{\rm d}x}$ must not be left in the text, but should be
% written ${{\rm d}y}/{{\rm d}x}$ or it should be displayed. Likewise write $n^{1/2}$ not $n^{\frac12}$.   Also $\displaystyle{a\choose{b}}$ and suchlike expressions
%  must not be left in the text. Equations
% involving lengthy expressions should, where possible, be avoided by introducing suitable
% notation.

% Symbols should not start sentences. Distinctive type, e.g. boldface, for matrices and
% vectors is not used in \Bka.\ Vectors are assumed to be column vectors, unless
% explicitly transposed. The use of an apostrophe to denote matrix or vector transposition should be avoided; it is preferable to write $A^\T$, $a^\T$.   Capital script letters may be used sparingly, typically to denote sets,
% but care should be taken as some are hard to distinguish.

% Please arrange brackets in the order $[\{(\ )\}]$, iterating as necessary,
%  and follow the usual conventions for $e$, exp, use of solidus, square root signs and so forth
% as in a recent issue. The sign $\sqrt{\hphantom{a}}$ is not used, and  the sign $\surd$ is used
% only sparingly; powers of complicated quantities should be represented as $(mnpq)^{a}$.

% Multiple overbars such as $\bar{\tilde{x}}$ must be avoided, as must $\widehat{ab}, (\widehat{a+b}), \widehat{\rm var}, \overline{ab},(\overline{a+b})$
% and symbols with underbars. Subscripts and superscripts, and second-order sub- and
% superscripts, should be aligned horizontally. Avoid sub- and superscripts of third, and greater, order.

% Please use: var($x$) not var $x$ or Var($x$); cov not Cov; pr for probability not Pr or P;
% tr not trace; $E(X)$ for expectation not $EX$ or ${\cal E}(X)$; $\log x$ not $\log_e x$ or $\ln x$; $r$th not $r$-th or $r^{\rm th}$.
% Please avoid: `$\cdot$' or `.' for product; $a/bc$, which should be written $a/(bc)$ or $a(bc)^{-1}$. Use the form $x_1,\ldots , x_n$ not $x_1, x_2,\ldots x_n$ and $\sum^{n}_{i=1}$ not $\sum^n_1$.  The typesetter will use a raised dot `$\cdot$' as a decimal point. Zeros precede decimal points: 0$\cdot$2 not $.2$.

% The use of `$\cdots$' and `$\ldots$' is  $\ldots$ in lists, such as $y_1,\ldots,y_n$, and $\cdots$ between binary operators,  giving $y_1+\cdots+y_n$.
% Ranges of integers are denoted $i=1,\ldots, n$, whereas $0\leq x\leq 1$ is used for ranges of real numbers.  %The support of density and other functions should always be given.

% \Bka\ deprecates the appearance of words in displayed equations, which should be formatted as
% \begin{equation}
% \label{one}
% \bar Y = n^{-1} \sum_{j=1}^n Y_j,\quad S^2 = \sum_{j=1}^n (Y_j-\bar Y)^2;
% \end{equation}
% note the punctuation and space between the expressions. Displays such as \eqref{one} should take no more space than necessary, being placed on a single line where possible.  Displayed mathematical expressions should be punctuated thus: indexed equations and similar quantities in text are formatted as
% $y_j = x_j^\T\beta + \v_j\ (j=1,\ldots, n)$, and are displayed as
% \[
% y_{ij} = x_j^\T\beta_i + \v_{ij}\quad (i=1,\ldots, m;\ j=1,\ldots, n).
% \]
% References to sequences of equations are (1)--(3), not (1--3).

% \subsection{Figures}

% Figures are a common source of delay during production, usually because elementary guidelines have not been respected.  General comments may be found in the document `RSSGraphs.pdf' enclosed with the \Bka\ formatting files, and more detail is given in standard references such as \citet{Cleveland:1993,Cleveland:1994} or \citet{Tufte:1983}.

% All the elements of a graph, including axis labels, should be large enough to be read easily, so the graph should be given a shape that will use the page space well. The use of large symbols, such as $\times$, for points should be avoided. If both axes of a panel show the same quantities, the panel should usually be square.  Many graphs are made using the statistical environment R \citep{R:2010}.  If so, they should be made at roughly the size at which they will appear in the journal.  Usually graphs reduced from A4 or US page sizes must be remade to ensure their legibility.

% Check that all the axes are labelled correctly and include units of measurement.  Axis labels should have the format `Difference of loglikelihoods': only the initial letter of the first word is upper-case.
% The numbers on the vertical axis should be parallel to the horizontal axis, and should be in the same font as the text; normally the change of font is left to the production process, but it is helpful if the numbers are placed horizontally.

% A panel should not contain an inset defining the line-types and symbols; this description should appear in the caption.

% Please submit figures in greyscale wherever possible.  \Bka\ publishes in colour where this is essential, but care should be taken to ensure that any colours chosen will be distinguishable on the cream paper used for the journal.

% Figures should be referred to consecutively by number. Use of the \LaTeX\ \verb+\label+ and \verb+\ref+ commands to refer to figures and tables helps to reduce errors and so is preferred.  Figure~\ref{fig1}  is a reference to a figure at the start of a sentence, whereas subsequent references are abbreviated, for example to Fig.~\ref{fig1}.

%  \begin{figure}
%  % The arguments in the next line are {height}{optional width}{used only by OUP for typesetting}[filename, in directory art]
% \figurebox{20pc}{25pc}{}[fig1.eps]
% % note that files may not be rotated
% \caption{A graph showing the truth (dot-dash), an estimate (dashes), another estimate (solid), and 95\% pointwise confidence limits (small dashes).}
% \label{fig1}
% \end{figure}

% \subsection{Tables}

% Tables should be referred to consecutively by number. Table is not abbreviated to Tab.

% Check that the arrangement makes effective use
% of the \Bka\ page. Layouts that have to be printed sideways should be avoided
% if possible. For this reason tables should not be more than 92 characters wide, including
% decimal points and brackets (1 character), and minus and other signs and spaces (at least 2
% characters).  Rules are not used in \Bka\ tables, which should be arranged to be clear without them.

% Often tables can be improved by multiplying all the entries by a power of ten, so that 0$\cdot$002 and
% 0$\cdot$02 become 2 and 20 respectively, for example; this will often both save space and convey the message of the table more effectively. Table~\ref{tablelabel} uses the definition \verb+\def~{\hphantom{0}}+ to insert invisible spaces into columns of the table; see the source code for this document.

% Very often tables containing results of Monte Carlo simulations use more digits than can be justified by the size of the simulation, and space can be gained and clarity improved by appropriate rounding.
% Standard errors or some other measure of precision should be given for Monte Carlo results.
% Often it suffices to give a phrase such as `The largest standard error for the results in column 2 is 0$\cdot$01.' in the caption to the table.

% \begin{table}
% \def~{\hphantom{0}}
% \tbl{Perceptions about racial groups in the U.S. population}{%
% \begin{tabular}{lcccc}
% %\\
%  \\
% & 2000 Census& \multicolumn{3}{c}{Mean percent estimated for U.S.} \\
% & percent of  & \multicolumn{3}{c}{population } \\
% & U.S. population & White Rs & Black Rs & Hispanic Rs \\[5pt]
% White & 75 & 59 & 56 & 60 \\
% Black & 12 & 30 & 38 & 40 \\
% Asian & ~4 & 16 & 21 & 30 \\
% American Indian & ~1 & 13 & 17 & 23 \\
% More than two races & ~2 & 41 & 48 & 50 \\
% Hispanic & 13 & 23 & 27 & 42
% \end{tabular}}
% \label{tablelabel}
% \begin{tabnote}
% U.S., United States of America; R, respondent.
% \end{tabnote}
% \end{table}

% \subsection{Captions to figures and tables}

% The caption to a figure should contain descriptions of lines and symbols used, but these should not be duplicated in the text, which should give the interpretation of the figure.  The figure should not contain an inset.  Verify that the caption and the graph agree, that every line and symbol is described correctly and that all lines or symbols in the graph are described in the caption.

% Any abbreviations used in the body of a table should be explained in a footnote to it.

% Figure captions always end with a full stop. The last sentence of a table title does not have a full stop.

% \subsection{References}

% References in the text should follow the current style used in \Bka.\ It is preferable to use BibTeX
% if possible, as in this guide.  In citing
% references use `First author et al.' if there are three or more authors. The list of references at
% the end should correspond to those in the text, and be in exactly current \Bka\
% form.

% References to books should be to the latest edition; a page, section or chapter number
% is nearly always necessary. References to books of papers should include title of book,
% editor(s), first and final page numbers of paper, where published and publisher.

% Complete lists of authors and editors should be given; in exceptional cases they may be abbreviated at the discretion of the editor.

% PhD theses, unpublished reports and articles can be referred to in the text, using a phrase such as `as shown in a 2009 Euphoric State University Department of Statistics PhD thesis by M.~Zapp' or `the proofs may be found in an unpublished 2003 technical report available from the first author', but should not be included in the References except where they have been accepted for publication, and unless they appear in a permanent repository such as arXiv; in this case the most recent version of the work is cited like a paper, e.g., \citet{Berrendero.etal:2015}.

% URLs for personal websites should be avoided as they become obsolete quickly and it is preferable to refer to the authors and institutions.  Technical details for published papers should be prepared as supplementary material, so that they remain permanently available. Likewise software should be submitted as supplementary material; it should be adequately documented, e.g., by including a README file to accompany R code

% \citet{Cox:1972} is an example of an active citation, and an example of a passive citation is  \citep{Hear:Holm:Step:quan:2006}.  The abbreviations for their journals should be noted.

% \subsection{Theorem-like environments}

% \Bka\ does not use \LaTeX\ list environments such as \texttt{itemise}, \texttt{description}, or \texttt{enumerate}.
% In this subsection we illustrate the use of theorem-like environments.

% \begin{definition}[Optional argument]
% This is a definition.
% \end{definition}

% \begin{assumption}[Another optional argument]
% \label{assumptionA}
% This is an assumption.
% \end{assumption}

% \begin{proposition}
% This is a proposition.
% \end{proposition}

% \begin{lemma}
% \label{lemma1}
% This lemma precedes a theorem.
% \end{lemma}

% \begin{proof}
% This is a proof of Lemma~\ref{lemma1}.  Perhaps it should be placed in the Appendix.
% \end{proof}

% \begin{theorem}
% \label{theorem1}
% This is a theorem.
% \end{theorem}

% Some text before we give the proof.

% \begin{proof}[of Theorem~\ref{theorem1}]
% The proof should be here.
% \end{proof}

% \begin{example}
% This is an example.
% \end{example}

% Some text before the next theorem.

% \begin{theorem}[Optional argument]
% Another important result.
% \end{theorem}

% \begin{corollary}
% This is a corollary.
% \end{corollary}

% \begin{remark}
% This is a remark.
% \end{remark}

% \begin{step}
% This is a step.
% \end{step}

% \begin{condition}
% This is a condition.
% \end{condition}


% \begin{property}
% This is a property.
% \end{property}

% \begin{restriction}
% This is a restriction.
% \end{restriction}

% \begin{algo}
% A simple algorithm.
% %\vspace*{-12pt}
% \begin{tabbing}
%   \qquad \enspace Set $s=0$\\
%   \qquad \enspace For $i=1$ to $i=n$ \\
%   \qquad \qquad Set $t=0$\\
%   \qquad \qquad For $j=1$ to $j=i$ \\\
%   \qquad \qquad\qquad  $t \leftarrow t + x_{ij}$ \\
% \qquad \qquad $s \leftarrow s + t$ \\
% \qquad \enspace Output $s$
% \end{tabbing}
% \end{algo}


% %\begin{algorithm}[!h]
% %\caption{A simple algorithm.} \label{al1}
% %\end{algorithm}


% \section{Discussion}

% This is the concluding part of the paper.  It is only needed if it contains new material.
% It  should not repeat the summary or reiterate the contents of the paper.

% \section*{Acknowledgement}
% Acknowledgements should appear after the body of the paper but before any appendices and be as brief as possible
% subject to politeness. Information, such as contract numbers, of no interest to readers, must
% be excluded.

% \section*{Supplementary material}
% \label{SM}
% Further material such as technical details, extended proofs, code, or additional  simulations, figures and examples may appear online, and should be briefly mentioned as Supplementary Material where appropriate.  Please submit any such content as a PDF file along with your paper, entitled `Supplementary material for Title-of-paper'.  After the acknowledgements, include a section `Supplementary material' in your paper, with the sentence `Supplementary material available at \Bka\ online includes $\ldots$', giving a brief indication of what is available.  However it should be possible to read and understand the paper without reading the supplementary material.

% Further instructions will be given when a paper is accepted.


% \appendix

% \appendixone
% \section*{Appendix 1}
% \subsection*{General}

% Any appendices appear after the acknowledgement but before the references, and have titles. If there is more than one appendix, then they are numbered,  as here Theorem~\ref{appthm1}.

% \begin{theorem}\label{appthm1}
% This is a rather dull theorem:
% \begin{equation}
% \label{A1}
% a + b = b + a;
% \end{equation}
% a little equation like this should only be displayed and labelled if it is referred to  elsewhere Lemma~{\rm \ref{lem:gdp-1}}.
% \end{theorem}

% \begin{lemma}\label{lem:gdp-1}
% If $\alpha_j > 2$, $\eta_j/\alpha_j = O(j^{-m})$ ($j=1, \ldots,
% \infty$) and $m > 1/2$, then $P_{l} (\Thetabb)= 1$.
% \end{lemma}



% \appendixtwo
% \section*{Appendix 2}
% \subsection*{Technical details}

% Often the appendices contain technical details of the main results.

% \begin{theorem}
% This is another theorem full of gory details.
% \end{theorem}

% \begin{lemma}\label{lem:gdp-2}
% If $\delta > 2$, $\rho > 0$, $\alpha_j(\delta) = \delta^j$ and
% $\eta_j(\rho) = \rho$ for $j = 1, \ldots, \infty$, then $P_{l}
% (\Thetabb) = 1$, where $P_{l}$ has density $p_{\mathrm{mgdP}}$ in
% (4) with hyperparameters $\alpha_j(\delta)$ and
% $\eta_j(\rho)$ ($j = 1, \ldots, \infty$). Furthermore, given
% $\epsilon > 0$, there exists a positive integer $k(p, \delta,
% \epsilon)= O\{\log^{-1} \delta \log ({p}/{\epsilon^2})\}$ for every
% $\Omega$ such that for all $r \geq k$, $\alpha_j(\delta)=\delta^j$,
% $\eta_j(\rho)=\rho$ ($j=1, \ldots, r$) and $\Omega^{r} = \Lambda^{r}
% {\Lambda^r}^{\T} + \Sigma$, we have that ${\rm pr}\{\Omega^{r} \mid
% d_{\infty}(\Omega, \Omega^{r}) < \epsilon\} > 1 - \epsilon$ where
% $d_{\infty}(A, B) = {\max}_{1 \leq i,j \leq p}|a_{ij} - b_{ij}|$.
% \end{lemma}

% \appendixthree
% \section*{Appendix 3}

% Often the appendices contain technical details of the main results:
% \begin{equation}
% \label{C1}
% a + b = c.
% \end{equation}

% \begin{remark}
% This is a remark concerning equations~\eqref{A1} and \eqref{C1}.
% \end{remark}

% \begin{lemma}\label{facstrong}
% The conditional density model $\mathcal{M}$ of $\S$\,3 is sequentially strongly convex with $H_{k}(p)\left( z\right)
% \equiv p\left( a_{k}\mid \overline{l}_{k},\overline{a}_{k-1}\right) $.
% \end{lemma}

% \bibliographystyle{biometrika}
% \bibliography{paper-ref}

% % \begin{thebibliography}{7}
% % \expandafter\ifx\csname natexlab\endcsname\relax\def\natexlab#1{#1}\fi

% % \bibitem[{Berrendero(2015)}]{Berrendero.etal:2015}
% % \textsc{ Berrendero, J. R., Cuevas, A. \& Torrecilla, J. L.} (2015).
% % \newblock On the use of reproducing kernel hilbert spaces in functional
% %   classification.
% % \newblock \textit{arXiv}: 1507.04398v3.

% % \bibitem[{Cleveland(1993)}]{Cleveland:1993}
% % \textsc{Cleveland, W.~S.} (1993).
% % \newblock \textit{{Vizualizing Data}}.
% % \newblock Summit: Hobart Press.

% % \bibitem[{Cleveland(1994)}]{Cleveland:1994}
% % \textsc{Cleveland, W.~S.} (1994).
% % \newblock \textit{{The Elements of Graphing Data}}.
% % \newblock Summit: Hobart Press, revised ed.

% % \bibitem[{Cox(1972)}]{Cox:1972}
% % \textsc{Cox, D.~R.} (1972).
% % \newblock {Regression models and life tables (with Discussion)}.
% % \newblock \textit{J. R. Statist. Soc. {\rm B}} \textbf{34}, 187--220.

% % \bibitem[{Heard et~al.(2006)Heard, Holmes \&
% %   Stephens}]{Hear:Holm:Step:quan:2006}
% % \textsc{Heard, N.~A.}, \textsc{Holmes, C.~C.} \& \textsc{Stephens, D.~A.}
% %   (2006).
% % \newblock A quantitative study of gene regulation involved in the immune
% %   response of {A}nopheline mosquitoes: {A}n application of {B}ayesian
% %   hierarchical clustering of curves.
% % \newblock \textit{J. Am. Statist. Assoc.} \textbf{101}, 18--29.

% % \bibitem[{{R Development Core Team}(2012)}]{R:2010}
% % \textsc{{R Development Core Team}} (2012).
% % \newblock \textit{R: A Language and Environment for Statistical Computing}.
% % \newblock Vienna, Austria: R Foundation for Statistical Computing.
% % \newblock {ISBN} 3-900051-07-0, http://www.R-project.org.

% % \bibitem[{Tufte(1983)}]{Tufte:1983}
% % \textsc{Tufte, E.~R.} (1983).
% % \newblock \textit{{The Visual Display of Quantitative Information}}.
% % \newblock Cheshire: Graphics Press.

% % \end{thebibliography}

\end{document}
